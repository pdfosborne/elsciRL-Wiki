\documentclass{article}
\usepackage{amsmath}
\usepackage{amsthm}
\usepackage{aliascnt}
\usepackage{biblatex}
\usepackage{graphicx}
\usepackage{hyperref}

\usepackage{listings}
\usepackage{color}

\definecolor{dkgreen}{rgb}{0,0.6,0}
\definecolor{gray}{rgb}{0.5,0.5,0.5}
\definecolor{mauve}{rgb}{0.58,0,0.82}

\lstset{frame=tb,
  language=python,
  aboveskip=3mm,
  belowskip=3mm,
  showstringspaces=false,
  columns=flexible,
  basicstyle={\small\ttfamily},
  numbers=none,
  numberstyle=\tiny\color{gray},
  keywordstyle=\color{blue},
  commentstyle=\color{dkgreen},
  stringstyle=\color{mauve},
  breaklines=true,
  breakatwhitespace=true,
  tabsize=3
}

\theoremstyle{plain}
\newtheorem{theorem}{Theorem}[section]

\renewcommand{\equationautorefname}{Equation}

\renewcommand{\sectionautorefname}{Section} % name for \autoref
\renewcommand{\subsectionautorefname}{Section} % name for \autoref
\renewcommand{\subsubsectionautorefname}{Section} % name for \autoref

\newaliascnt{proposition}{theorem}% alias counter "<newTh>"
\newtheorem{proposition}[proposition]{Proposition}
\aliascntresetthe{proposition}
\providecommand*{\propositionautorefname}{Proposition} % name for \autoref

\newaliascnt{corollary}{theorem}% alias counter "<newTh>"
\newtheorem{corollary}[corollary]{Corollary}
\aliascntresetthe{corollary}
\providecommand*{\corollaryautorefname}{Corollary} % name for \autoref

\newaliascnt{lemma}{theorem}% alias counter "<newTh>"
\newtheorem{lemma}[lemma]{Lemma}
\aliascntresetthe{lemma}
\providecommand*{\lemmaautorefname}{Lemma} % name for \autoref

\theoremstyle{definition}
\newtheorem{definition}{Definition}[section]
\newtheorem{example}{Example}

\theoremstyle{remark}
\newtheorem{rmk}{Remark}[section]
\newtheorem{fact}[rmk]{Fact}
\newtheorem*{rmk*}{Remark}
\newtheorem*{fact*}{Fact}

\newenvironment{remark}
{\pushQED{\qed}\renewcommand{\qedsymbol}{$\diamond$}\rmk}
{\popQED\endrmk}

\providecommand*{\remarkautorefname}{Remark} % name for \autoref
\providecommand*{\rmkautorefname}{Remark} % name for \autoref
\providecommand*{\definitionautorefname}{Definition} % name for \autoref
\addbibresource{bibliography.bib}
\title{}
\author{Philip Osborne}
\begin{document}
\maketitle
\section{Tabular vs Neural Agent Architectures}
\label{loc:tabular_vs_neural_agent_architectures}
Any reinforcement learning agent must include a method for language understanding. 

An agent which stores every state-action value in a large lookup table is known as a \emph{tabular} agent and has the limitation that every state is considered unique. A tabular agent has no intrinsic methodology for leveraging the contextual information of language.  ^1c5f81

Therefore, \emph{neural} (a.k.a deep) agents \cite[^1]{mnih2013PlayingAtariDeep} are introduced as they can transfer knowledge between similar states. 

To achieve this, a Deep-RL agent`s architecture consists of two core components: 
1) **a state encoder** an encoder for transforming a state into a numeric form (typically vector)
2) **an action selection** method which enacts the agent's policy.

---
1: [[Wiki/References/Academic Papers/ArXiv/mnih2013PlayingAtariDeep|\cite{mnih2013PlayingAtariDeep}]

[^1]: \hyperref[loc:wiki/references/academic_papers/arxiv/mnih2013playingatarideep.statement]{mnih2013PlayingAtariDeep}
\printbibliography
\end{document}