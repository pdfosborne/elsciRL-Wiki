\documentclass{article}
\usepackage{amsmath}
\usepackage{amsthm}
\usepackage{aliascnt}
\usepackage{biblatex}
\usepackage{graphicx}
\usepackage{hyperref}

\usepackage{listings}
\usepackage{color}

\definecolor{dkgreen}{rgb}{0,0.6,0}
\definecolor{gray}{rgb}{0.5,0.5,0.5}
\definecolor{mauve}{rgb}{0.58,0,0.82}

\lstset{frame=tb,
  language=python,
  aboveskip=3mm,
  belowskip=3mm,
  showstringspaces=false,
  columns=flexible,
  basicstyle={\small\ttfamily},
  numbers=none,
  numberstyle=\tiny\color{gray},
  keywordstyle=\color{blue},
  commentstyle=\color{dkgreen},
  stringstyle=\color{mauve},
  breaklines=true,
  breakatwhitespace=true,
  tabsize=3
}

\theoremstyle{plain}
\newtheorem{theorem}{Theorem}[section]

\renewcommand{\equationautorefname}{Equation}

\renewcommand{\sectionautorefname}{Section} % name for \autoref
\renewcommand{\subsectionautorefname}{Section} % name for \autoref
\renewcommand{\subsubsectionautorefname}{Section} % name for \autoref

\newaliascnt{proposition}{theorem}% alias counter "<newTh>"
\newtheorem{proposition}[proposition]{Proposition}
\aliascntresetthe{proposition}
\providecommand*{\propositionautorefname}{Proposition} % name for \autoref

\newaliascnt{corollary}{theorem}% alias counter "<newTh>"
\newtheorem{corollary}[corollary]{Corollary}
\aliascntresetthe{corollary}
\providecommand*{\corollaryautorefname}{Corollary} % name for \autoref

\newaliascnt{lemma}{theorem}% alias counter "<newTh>"
\newtheorem{lemma}[lemma]{Lemma}
\aliascntresetthe{lemma}
\providecommand*{\lemmaautorefname}{Lemma} % name for \autoref

\theoremstyle{definition}
\newtheorem{definition}{Definition}[section]
\newtheorem{example}{Example}

\theoremstyle{remark}
\newtheorem{rmk}{Remark}[section]
\newtheorem{fact}[rmk]{Fact}
\newtheorem*{rmk*}{Remark}
\newtheorem*{fact*}{Fact}

\newenvironment{remark}
{\pushQED{\qed}\renewcommand{\qedsymbol}{$\diamond$}\rmk}
{\popQED\endrmk}

\providecommand*{\remarkautorefname}{Remark} % name for \autoref
\providecommand*{\rmkautorefname}{Remark} % name for \autoref
\providecommand*{\definitionautorefname}{Definition} % name for \autoref
\addbibresource{bibliography.bib}
\title{}
\author{Philip Osborne}
\begin{document}
\maketitle
\section{Q-Learning}
\label{loc:q:learning}
An agent must incorporate the long-term outcomes of actions into the calculations. A well known method for achieving this is Q-learning and defined by \cite{sutton2018ReinforcementLearningIntroduction} [\cite{sutton2018ReinforcementLearningIntroduction}(Wiki/References/Books/sutton2018ReinforcementLearningIntroduction) whereby calculations are updated based on the transitions from the current state-action pair to the next state.

Formally, the value of state-action pairs, $Q(s,a)$, are updated using the following calculation in equation \autoref{eq:q_learn_update_rule} after reaching every non-terminal state. If the next state is a terminal state, then this is fixed to $Q(s^\prime,a^\prime)=0$ and often a large reward is provided depending on the outcome. Over time, the numeric results of the long-term outcome propagates backwards to the earliest states in an episode such that the agent can make immediate actions that are not going to cause long-term issues. 
\subsubsection{Q-Learning Update Rule:}
\label{loc:q:learning.q:learning_update_rule:}
\begin{equation}
\label{eq:q_learn_update_rule}
Q^{new}(s,a)\leftarrow Q(s,a) + \alpha {\bigg (} r + \gamma \max_{a'}Q(s',a') - Q(s,a) {\bigg )}
\end{equation}

where:
\begin{itemize}
\item $Q(s,a)$ is the value of state-action pair $s$,
\item $\alpha$ is the learning rate parameter,
\item $r$ is the immediate reward,
\item $\gamma$ is the discount factor parameter and,
\item $Q(s', a')$ is the value of action-pair in the next state taking the best known action.
\end{itemize}

The method is defined as \emph{off-policy} as the calculation is based on the maximum possible value that could be obtained in the next state rather than one based on an action selection policy. 

---
1: [[Wiki/References/Books/sutton2018ReinforcementLearningIntroduction|\cite{sutton2018ReinforcementLearningIntroduction}]
\printbibliography
\end{document}